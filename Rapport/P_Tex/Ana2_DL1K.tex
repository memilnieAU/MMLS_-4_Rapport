DL 2. orden for 1k$\Omega$

1. Vores ligning \ref{KVL2_done} står allerede på den rigtige form, så ud fra denne ligning kan vi finde vores konstanter a, b, c og vores funktion f(t)

\begin{center}
\begin{minipage}{.2\linewidth}
a = L $\cdot$ C
\end{minipage}
\begin{minipage}{.2\linewidth}
b = R $\cdot$ C
\end{minipage}
\begin{minipage}{.2\linewidth}
c = 1
\end{minipage}
\begin{minipage}{.2\linewidth}
f(t) = $V_{in}(t)$
\end{minipage}
\end{center}

2. Find den homogene løsning: 
\begin{equation}
	L\cdot C\cdot \dfrac{d^{2}}{dt^{2}}\cdot V_{out}  \left(t\right) +R\cdot C\cdot \dfrac{d}{dt}\ V_{out}  \left(t\right) +V_{out} = 0
\end{equation}

2.1 opskriv karakterligningen:
\begin{equation}
	L\cdot C\cdot k^{2}+R\cdot C\cdot k+c = 0
\end{equation}

2.2 Løs karakterligningen

\begin{center}
\begin{minipage}{.2\linewidth}
R=1k$\Omega$
\end{minipage}
\begin{minipage}{.2\linewidth}
L = 1 mH
\end{minipage}
\begin{minipage}{.2\linewidth}
C = 1 nF
\end{minipage}
\end{center}

\begin{equation}
	k_{1} = \dfrac{ -R\cdot C+ \sqrt{ \left(R\cdot C\right) ^{2}-4\cdot L\cdot C\cdot 1} }{2\cdot L\cdot C}\rightarrow{simplify} 
	\dfrac{\dfrac{ \sqrt{k^{\Omega}\cdot nF^{2}-4\cdot nF\cdot mH} }{2}-\dfrac{k \Omega\cdot nF}{2}}{nF\cdot mH}
\end{equation}

\begin{equation}
	k_{1} =  \left( - \left(5\cdot 10^{5}\right) +8.66\cdot j\right) \ \dfrac{1}{s}
\end{equation}

\begin{equation}
	k_{2} = \dfrac{ -R\cdot C- \sqrt{ \left(R\cdot C\right) ^{2}-4\cdot L\cdot C\cdot 1} }{2\cdot L\cdot C} \rightarrow{simplify}
	 -\dfrac{k\cdot \Omega+ \sqrt{k^{\Omega}\cdot nF^{2}-4\cdot nF\cdot mH} }{2\cdot nF\cdot mH}
\end{equation}

\begin{equation}
	k_{2} =  \left( - \left(5\cdot 10^{5}\right) -8.66\cdot j\right) \ \dfrac{1}{s}
\end{equation} 

2.3 Homogen løsning $V_{out}{10}h(x)$

\begin{equation}
D = b^2 - 4\cdot a\cdot c = 0
\end{equation}

\begin{equation}
	D =  \left(1\ k\cdot 1\ nF\right) ^{2}-4\cdot 1\ mH\cdot 1\ nF\cdot 1 =  -3\cdot 10^{ -12}\ s^{2}
\end{equation}

\begin{equation}
	R_{cr} = 2\cdot  \sqrt{\dfrac{L}{C}}  = 2\ k\Omega
\end{equation}

Da $k_2$ er den kompleks konjugerede af $k_1$, samt den kritiske værdi er større end modstanden R = 1 k$\Omega$. \\
Skal løsningen stå på følgende form:

\begin{equation}
y_h(x)=e^{\alpha x} \cdot (A \cdot cos (\beta x) + B\cdot sin(\beta x))
\end{equation}

\begin{center}
\begin{minipage}{.2\linewidth}
$\alpha =  -5\cdot 10^{5}\ \dfrac{1}{s}$
\end{minipage}
\begin{minipage}{.2\linewidth}
$\beta = 8.66\cdot 10^{5}\ \dfrac{1}{s}$
\end{minipage}
\end{center}

Herefter indsættes værdierne for $\alpha$ og $\beta$

\begin{equation}
	V_{out\_1\_h} \left(t\right)  = e^{\alpha\cdot t}\cdot  \left(A\cdot cos  \left(\beta\cdot t\right) +B\cdot \sin  \left(\beta\cdot t\right) \right)  \xrightarrow{}
	\label{Homegen1K}
	\end{equation}
	\begin{center}
	$ e^{ -\dfrac{500000\cdot t}{s}}\cdot  \left(A\cdot cos  \left(\dfrac{866000\cdot t}{s}\right) +B\cdot \sin  \left(\dfrac{866000\cdot t}{s}\right) \right)$
	
	\end{center}



3. Find partikulær løsning
\begin{equation}
	V_{out\_10}  \left(t\right)  = V_{in}  \left(t\right)  = V_{0} = 5\ V
\end{equation}

3.1 gæt en løsning \\

Kvalificeret bud
\begin{center}
\begin{minipage}{.2\linewidth}$V_{out\_1\_p}  \left(t\right)  = \alpha$
\end{minipage}
\begin{minipage}{.2\linewidth}
$V$'$_{out\_1\_p}  \left(t\right) = 0$
\end{minipage}
\begin{minipage}{.2\linewidth}
$V$''$_{out\_1\_p}  \left(t\right)  = 0$
\end{minipage}
\end{center}

VS: $L\cdot C\cdot 0 + R\cdot C\cdot 0 + \alpha$ $<=>$ $\alpha$ \\

HS: $V_0$ = 5 V \\

VS=HS $\alpha$=5 V $<=>$ $V_{out\_1\_h}$ = 5 V \\

4. Fuldstændig løsning ($y(x) = y_h(x)+y_p(x)$)

\begin{equation}
	V_{out\_1\_h}  \left(t\right)  = e^{\dfrac{ -500000\cdot t}{s}}\cdot  \left(A\cdot cos  \left(866000\cdot t\right) +B\cdot \sin  \left(866000\cdot t\right) \right) 
\end{equation}

\begin{equation}
	V_{out\_1\_p} \left(t\right)  = 5\ V
	\label{Partikulær1K}
\end{equation}

\begin{equation}
	V_{out\_1} \left(t\right)  = V_{out\_1\_h}  \left(t\right) +V_{out\_1\_p}  \left(t\right)  \xrightarrow{} 5\cdot V+e^{ -\dfrac{500000\cdot t}{s}}\cdot  \left(A\cdot cos  \left(\dfrac{866000\cdot t}{s}\right) +B\cdot \sin  \left(\dfrac{866000\cdot t}{s}\right) \right) 
\end{equation}

5. Specifik løsning (bestem betingelser)

1. Kondensatoren kan ikke oplades øjeblikkelig, så til tiden 0 er spændingen = 0: \\

\begin{equation}
	V_{out\_1}  \left(0\right)  = 0 \xrightarrow{} A+5\cdot V = 0
\end{equation}

\begin{equation}
	V_{out\_10}  \left(0\right)  = 0 \xrightarrow{} A+B+5\cdot V = 0
\end{equation}

2. En spole kan ikke acceptere øjeblikkelige strømændringer, derfor vil strømmen være 0 til tiden 0:

i(0) = 0 $<=>$ $dV_{out}{1}(0)=0$

\begin{equation}
 dV_{out\_1} \left(t\right) = \dfrac{d}{dt}V_{out}{1}(t)
\end{equation}
 
\begin{equation}
	dV_{out\_1}  \left(0\right)  = 0 \xrightarrow{} \dfrac{866000\cdot B}{s}-\dfrac{500000\cdot A}{s} = 0
\end{equation}

\begin{equation}
	\begin{bmatrix}
		A &B \\
	\end{bmatrix}
	 = 
	\begin{bmatrix}
		A+5\cdot V = 0 \\
		866000\cdot B-500000\cdot A = 0 \\
	\end{bmatrix}
	 \xrightarrow{solve, A, B} 
	\begin{bmatrix}
		 -5\cdot V & -2.8868360277136\cdot V \\
	\end{bmatrix}
\end{equation}

A = -5 V
B = -2.887 V \\

6. Tjek

\begin{equation}
	V_{out\_1} \left(t\right)  = 5\cdot V+e^{ -\dfrac{500000\cdot t}{s}}\cdot  \left( -5\ V\cdot cos  \left(\dfrac{866000\cdot t}{s}\right) -2.89\cdot V\cdot \sin  \left(\dfrac{866000\cdot t}{s}\right) \right)
	\label{specifik1K} 
\end{equation}

Omskrives ved hjælp af kompleks symbolsk metode: \\

\begin{equation}
	V_{out\_1}  \left(t\right)  =  -5\ V\cdot cos  \left(\dfrac{866000\cdot t}{s}\right) -2.89\cdot V\cdot \sin  \left(\dfrac{866000\cdot t}{s}\right)  = 
	\end{equation}
\begin{center}
$ -5\ V\cdot cos  \left(\dfrac{866000\cdot t}{s}\right) -2.89\cdot V\cdot \sin  \left(\dfrac{866000\cdot t}{s}-\dfrac{\pi}{2}\right) $
\end{center}



