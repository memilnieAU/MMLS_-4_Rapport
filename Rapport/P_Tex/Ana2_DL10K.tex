\subsubsection*{2. Løs differentialligningen med hensyn til Vout }
 Løs differentialligningen med hensyn til $V_{out}$ for $ 0\leq t<\infty$ med hhv. R = 10k$\Omega$ og \\ R = 1k$\Omega$ \\

Vi har brugt brug af løsningsprotokolen for en 2. orden DL.

DL 2. orden for 10k$\omega$

1. Vores ligning \ref{KVL2_done} står allerede på den rigtige form, så ud fra denne ligning kan vi finde vores konstanter a, b, c og vores funktion f(t)

\begin{center}
\begin{minipage}{.2\linewidth}
a = L $\cdot$ C
\end{minipage}
\begin{minipage}{.2\linewidth}
b = R $\cdot$ C
\end{minipage}
\begin{minipage}{.2\linewidth}
c = 1
\end{minipage}
\begin{minipage}{.2\linewidth}
f(t) = $V_{in}(t)$
\end{minipage}
\end{center}

2. Find den homogene løsning: 
\begin{equation}
	L\cdot C\cdot \dfrac{d^{2}}{dt^{2}}\cdot V_{out}  \left(t\right) +R\cdot C\cdot \dfrac{d}{dt}\ V_{out}  \left(t\right) +V_{out} = 0
\end{equation}

2.1 opskriv karakterligningen:
\begin{equation}
	L\cdot C\cdot k^{2}+R\cdot C\cdot k+c = 0
\end{equation}

2.2 Løs karakterligningen

\begin{center}
\begin{minipage}{.2\linewidth}
R=10k$\Omega$
\end{minipage}
\begin{minipage}{.2\linewidth}
L = 1 mH
\end{minipage}
\begin{minipage}{.2\linewidth}
C = 1 nF
\end{minipage}
\end{center}

\begin{equation}
	k_{1} = \dfrac{ -R\cdot C+ \sqrt{ \left(R\cdot C\right) ^{2}-4\cdot L\cdot C\cdot 1} }{2\cdot L\cdot C} \rightarrow 
	\dfrac{2\cdot  \sqrt{25\cdot k^{\Omega}\cdot nF^{2}-nF\cdot mH} -10\cdot k\cdot nF}{2\cdot nF\cdot mH}
\end{equation}

$k_1$ = $\left( -1.01\cdot 10^{5}\right) \ \dfrac{1}{s}$

\begin{equation}
	k_{2} = \dfrac{ -R\cdot C- \sqrt{ \left(R\cdot C\right) ^{2}-4\cdot L\cdot C\cdot 1} }{2\cdot L\cdot C} \rightarrow 
	 -\dfrac{10\cdot k\cdot nF+2\cdot  \sqrt{25\cdot k^{\Omega}\cdot nF^{2}-nF\cdot mH} }{2\cdot nF\cdot mH}
\end{equation}

$k_{2} =  \left( -9.899\cdot 10^{6}\right) \ \dfrac{1}{s}$ \\

2.3 Homogen løsning $V_{out}{10}h(x)$

\begin{equation}
D = b^2 - 4\cdot a\cdot c = 0
\end{equation}

\begin{equation}
	D =  \left(R\cdot C\right) ^{2}-4\cdot L\cdot C =  \left(9.6\cdot 10^{ -11}\right) \ s^{2}
\end{equation}

Diskriminanten er altså større end 0, derfor skal den homogene løsning stå på formen

\begin{equation}
	y_h  \left(x\right)  = A\cdot e^{k_{1}\cdot t}+B\cdot e^{k_{2}\cdot t}
	\label{homogen_10k}
\end{equation}

Herefter indsættes værdierne for $k_1$ og $k_2$

\begin{equation}
	V_{out\_10\_h} \left(t\right)  = A\cdot e^{k_{1}\cdot t}+B\cdot e^{k_{2}\cdot t} \xrightarrow{} B\cdot e^{ -\dfrac{9.899\cdot 10^{6}\cdot t}{s}}+A\cdot e^{ -\dfrac{101000\cdot t}{s}}
	\label{homogen10K}
\end{equation}

3. Find partikulær løsning
\begin{equation}
	V_{out\_10}  \left(t\right)  = V_{in}  \left(t\right)  = V_{0} = 5\ V
\end{equation}

3.1 gæt en løsning \\

Kvalificeret bud
\begin{center}
\begin{minipage}{.2\linewidth}$V_{out\_10\_p}  \left(t\right)  = \alpha$
\end{minipage}
\begin{minipage}{.2\linewidth}
$V$'$_{out\_10\_p}  \left(t\right) = 0$
\end{minipage}
\begin{minipage}{.2\linewidth}
$V$''$_{out\_10\_p}  \left(t\right)  = 0$
\end{minipage}
\end{center}

VS: $L\cdot C\cdot 0 + R\cdot C\cdot 0 + \alpha$ $<=>$ $\alpha$ \\

HS: $V_0$ = 5 V \\

VS=HS \begin{equation}
\alpha=5 V <=> V_{out\_10\_p} = 5 V 
\end{equation} 
4. Fuldstændig løsning ($y(x) = y_h(x)+y_p(x)$)

\begin{equation}
	V_{out\_10\_h} \left(t\right)  = B\cdot e^{ -\dfrac{9.899\cdot 10^{6}\cdot t}{s}}+A\cdot e^{ -\dfrac{101000\cdot t}{s}}
\end{equation}

\begin{equation}
	V_{out\_10\_p} \left(t\right)  = 5\ V
	\label{partikulær10K}
\end{equation}

\begin{equation}
	V_{out\_10} \left(t\right)  = V_{out\_10\_h}  \left(t\right) +V_{out\_10\_p}  \left(t\right)  \xrightarrow{} 5\cdot V+B\cdot e^{ -\dfrac{9.899\cdot 10^{6}\cdot t}{s}}+A\cdot e^{ -\dfrac{101000\cdot t}{s}}
\end{equation} 

5. Specifik løsning (bestem betingelser)

Grundet spolen vil spændingen og strømmen ikke stige med det samme, når der er input. Derfor vil strømmen og spændingen være 0 til tiden 0. \\

1. \begin{equation}
V_{out\_10}  \left(0\right)  = 0
\end{equation}

\begin{equation}
	V_{out\_10}  \left(0\right)  = 0 \xrightarrow{} A+B+5\cdot V = 0
\end{equation}

2. \begin{equation}
 dV_{out\_10} \left(t\right) = \dfrac{d}{dt}V_{out}{10}(t)
\end{equation}
 
\begin{equation}
	dV_{out\_10}  \left(0\right)  = 0 \xrightarrow{}  -\dfrac{9.899\cdot 10^{6}\cdot B}{s}-\dfrac{101000\cdot A}{s} = 0
\end{equation}

\begin{equation}
	\begin{bmatrix}
		A &B \\
	\end{bmatrix}
	 = 
	\begin{bmatrix}
		A+B+5\cdot V = 0 \\
		 -\dfrac{9.899\cdot 10^{6}\cdot B}{s}-\dfrac{101000\cdot A}{s} = 0 \\
	\end{bmatrix}
	 \xrightarrow{solve, A, B, float, 3} 
	\begin{bmatrix}
		 -5.05\cdot V &0.0515\cdot V \\
	\end{bmatrix}
\end{equation}

A = -5.05 V
B = 0.052 V \\

6. Tjek

\begin{equation}
	V_{out\_10} \left(t\right)  = 5\cdot V+0.052\cdot V\cdot e^{ -\dfrac{9.899\cdot 10^{6}\cdot t}{s}}-5.05\cdot V\cdot e^{ -\dfrac{101000\cdot t}{s}}
	\label{specifik10k}
\end{equation}

 